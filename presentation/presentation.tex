% -*- mode: latex-mode; -*-

\documentclass{beamer}

%%%%%%%%%%%%%%%%%%%%%%%%%%%%%%%%%%%%%%%%%%%%%%%%%%%%%%%%%%%
% Source: https://github.com/pblottiere/dark-beamer-theme %
%%%%%%%%%%%%%%%%%%%%%%%%%%%%%%%%%%%%%%%%%%%%%%%%%%%%%%%%%%%

\usetheme{dbt}

%%%%%%%%%%%%%%%%%%%%%%%%%%%%%%%%%%%%%%%%%%%%%%%%%
% Source: https://tex.stackexchange.com/a/74127 %
%%%%%%%%%%%%%%%%%%%%%%%%%%%%%%%%%%%%%%%%%%%%%%%%%


\usepackage{tcolorbox}

\tcbset{%
  colback=black!95,%
  colupper=white,%
  colframe=white!40!black,%
  sharp corners%
}

%%%%%%%%%%%%%%%%%%%%%%%%%%%%%
% Deduction Rule Formatting %
%%%%%%%%%%%%%%%%%%%%%%%%%%%%%

\usepackage{mathpartir}

%%%%%%%%%%%%%%%%%%%%%%%%%%%%%%%%%%%%%%%%%%%%%%%%%%%%%%%%%%%%%%%%%%%%%%
% Source: https://isabelle.in.tum.de/community/Generate_TeX_Snippets %
%%%%%%%%%%%%%%%%%%%%%%%%%%%%%%%%%%%%%%%%%%%%%%%%%%%%%%%%%%%%%%%%%%%%%%

\usepackage{generated/document/comment}%
\usepackage{generated/document/isabelle}%
\usepackage{generated/document/isabellesym}%
\usepackage{generated/document/isabelletags}%
\usepackage{ifthen}%

\newcommand{\DefineSnippet}[2]{%
  \expandafter\newcommand\csname snippet--#1\endcsname{ \kern-1ex #2
    \kern-1.5ex }}

\newcommand{\Snippet}[1]{%
  \ifcsname snippet--#1\endcsname {\csname snippet--#1\endcsname}
  \else +++++++ERROR: Snippet ``#1'' not defined+++++++ \fi}

\input{generated/snippets}

% https://www.cl.cam.ac.uk/research/hvg/Isabelle/dist/Isabelle2017/doc/sugar.pdf
\usepackage{mathpartir}

%%%%%%%%
% Main %
%%%%%%%%

\title{\texttt{Cont r} is not a Comonad} \author{Matthew Doty} \date{}

\begin{document}

{\setbeamertemplate{footline}{} \frame{\titlepage}}


\hypersetup{colorlinks=false}

\section*{Outline}%

\hypersetup{colorlinks}

\begin{frame}[shrink=35]{\insertsectionhead}%
  \ \
  \tableofcontents %
\end{frame}

\section{Stuff You Can't Program}
\begin{frame}[plain]
  \vfill \centering
  \begin{beamercolorbox}[sep=8pt,center,shadow=true,rounded=true]{title}
    \usebeamerfont{title}\insertsectionhead\par%
  \end{beamercolorbox}
  \vfill
\end{frame}

\subsection{Halting Problem}
\begin{frame}[plain]{\insertsectionhead\ \textemdash\
    \insertsubsectionhead}

  The most famous negative result in computer science is due to Alan
  Turing \cite{turingComputableNumbersApplication1937}\\~\\

  \begin{proposition}
    There is no Turing machine that can compute for all programs
    $\ulcorner T\, \urcorner$ if they will halt or not on $x$.\\~\\
  \end{proposition}

  $\ulcorner T\, \urcorner$ is the code for Turing machine $T$
  given a \emph{universal} Turing machine $U$

\end{frame}

\subsection{Other Impossible Tasks}
\begin{frame}[plain]{\insertsectionhead\ \textemdash\
    \insertsubsectionhead}

  \begin{description}
  \item[G\"odel] Can't compute if there's a proof in Peano arithmetic
    for an arbitrary $\phi$
    \cite{goedelUeberFormalUnentscheidbare1931}
  \item[Kolmogorov] Can't make the perfect compression algorithm
    \cite{kolmogorovThreeApproachesQuantitative1968}
  \item[Soare] Can't compute if a program halts on every input
    \cite{soareTuringComputabilityTheory2016}
  \end{description}

\end{frame}

\subsection{Ever More Impossible}
\begin{frame}[allowframebreaks]{\insertsectionhead\ \textemdash\
    \insertsubsectionhead}

  If you \emph{could} compute for any program if would halt on an
  input, you could compute for any $\phi$ if it was provable in
  arithmetic or not.\\~\\

  However, you could not compute if a program halts \emph{on every
    input}.  This is even more \emph{uncomputable}!\\~\\

\framebreak

  The relation ``\emph{if I could compute $x$, I could compute $y$}''
  (called \emph{Turing reducibility}) creates classes of algorithms
  like complexity theory.\\~\\

  For every problem $P$, we can make a $Q$ where even if we could
  solve $P$ then $Q$ would \emph{still} be impossible.\\~\\

  There are an infinite number of seperable classes in higher
  complexity theory under Turing reducibility.

\end{frame}

\subsection{Another \emph{Type} of Impossibility}
\begin{frame}[plain]{\insertsectionhead\ \textemdash\
    \insertsubsectionhead}

  All the impossibility theorems mentioned involve some kind of \emph{diagonal argument} or another.\\~\\

  Types were invented by Bertrand Russell to avoid the paradox he
  uncovered in Frege's \emph{Die Grundlagen der Arithmetik} \cite{russelLetterFrege1981}. \\~\\

  A century later we can use types as an alternate way of proving
  programs don't exist.
\end{frame}

\section{The Continuation Monad}
\begin{frame}[plain]
  \vfill \centering
  \begin{beamercolorbox}[sep=8pt,center,shadow=true,rounded=true]{title}
    \usebeamerfont{title}\insertsectionhead\par%
  \end{beamercolorbox}
  \vfill
\end{frame}

\subsection{Definition}
\begin{frame}[fragile]{\insertsectionhead\ \textemdash\
    \insertsubsectionhead}
  \begin{lstlisting}
newtype Cont r a
  = Cont {runCont :: (a -> r) -> r}

instance Monad (Cont r) where
  return x = Cont ($ x)
  s >>= f =
    Cont $ \c ->
      runCont s $ \x ->
        runCont (f x) c
  \end{lstlisting}
\end{frame}

\subsection{Example Usage}
\begin{frame}[fragile]{\insertsectionhead\ \textemdash\
    \insertsubsectionhead}
  |Cont r| can be used for a non-local exit \\~\\

  This is called an \emph{escape continuation}; see the \href{https://hackage.haskell.org/package/mtl-2.2.2/docs/Control-Monad-Cont.html\#g:5}{\texttt{mtl}} package for more details\\~\\

\begin{lstlisting}
whatsYourName :: String -> String
whatsYourName name = (`runCont` id) $ do
  response <- callCC $ \exit -> do
    when (null name) $
      exit "Must have a name!"
    return $ "Welcome, " ++ name ++ "!"
  return response
\end{lstlisting}
\end{frame}

\section{Combinatory Logic}
\begin{frame}[plain]
  \vfill \centering
  \begin{beamercolorbox}[sep=8pt,center,shadow=true,rounded=true]{title}
    \usebeamerfont{title}\insertsectionhead\par%
  \end{beamercolorbox}
  \vfill
\end{frame}

\subsection{History}
\begin{frame}[plain]{\insertsectionhead\ \textemdash\
    \insertsubsectionhead}
Topic of Haskell Curry's PhD thesis in 1930 \cite{curryGrundlagenKombinatorischenLogik1930}\\~\\

Based on earlier work by Moses Sch\"onfinkel in 1924 \cite{schoenfinkelUeberBausteineMathematischen1924}\\~\\

The presentation here has been formalized in Isabelle/HOL
\end{frame}

\subsection{Syntax}
\begin{frame}[plain]{\insertsectionhead\ \textemdash\
    \insertsubsectionhead}
  \Snippet{SKCombDef}
\end{frame}

\subsection{Simple Typing}
\begin{frame}[fragile]{\insertsectionhead\ \textemdash\
    \insertsubsectionhead}
  Simple typing is achieved for the pure $S$ and $K$ combinators using
  the following inductively defined predicate
  \Snippet{simple_type_predicate}\\~\\

  \begin{center}
    \Snippet{K_type}\\[2ex]

    \Snippet{S_type}\\[4ex]

    \Snippet{Application_type}
  \end{center}
\end{frame}

\subsection{Lambda-Abstraction}
\begin{frame}[allowframebreaks]{\insertsectionhead\ \textemdash\
    \insertsubsectionhead}
  The $\lambda$-calculus can be embedded in combinatory logic.

  The embedding here is due to David Turner
  \cite{turnerAnotherAlgorithmBracket1979}. \\~\\

  \begin{center}
    \Snippet{free_variables}
  \end{center}

  \framebreak

  \begin{center}
    \Snippet{Turner_Abstraction}
  \end{center}

\end{frame}

\section{Kripke Semantics}
\begin{frame}[plain]
  \vfill \centering
  \begin{beamercolorbox}[sep=8pt,center,shadow=true,rounded=true]{title}
    \usebeamerfont{title}\insertsectionhead\par%
  \end{beamercolorbox}
  \vfill
\end{frame}

\subsection{History}
\begin{frame}[allowframebreaks]{\insertsectionhead\ \textemdash\
    \insertsubsectionhead}

\emph{Kripke Semantics} refers to possible world semantics given a \emph{transition} relation.\\~\\

Credited to the mathematician Saul Kripke, who invented these models for logic while in high school in 1959 \cite{kripkeCompletenessTheoremModal1959}\\~\\

\framebreak

In the 60s \& 70s, Hoare \cite{hoareAxiomaticBasisComputer1969} and Pratt \cite{prattSemanticalConsiderationsFloydHoare1976} adapted Kripke models to \emph{Labeled Transition Systems} in order to generalize Kleene's \emph{regular expressions}. \\~\\

Also in the 70s, Pnueli used Kripke semantics for \emph{Linear Temporal Logic} (LTL), which he proposed for formal verification \cite{pnueliTemporalLogicPrograms1977}.  This inspired Leslie Lamport to ultimately create \emph{TLA+} \cite{lamportSpecifyingConcurrentSystems1999}. \\~\\

\framebreak

We are going to use Kripke semantics for \emph{intuitionistic logic}. This is the logic of simple types according to the Curry-Howard correspondence.

In a sense, Kripke semantics can be seen as the \emph{dual} to Combinatory logic.

\end{frame}

\subsection{Data Structure}
\begin{frame}[plain]{\insertsectionhead\ \textemdash\
    \insertsubsectionhead}
  \begin{center}
  \Snippet{Kripke_Model}
  \end{center}
\end{frame}

\subsection{Model Theory}
\begin{frame}[plain]{\insertsectionhead\ \textemdash\
    \insertsubsectionhead}
  Define the \emph{Tarski Truth Predicate} \Snippet{Tarski_Truth} inductively as follows:\\~\\

  \begin{center}
    \Snippet{Intuitionistic_Kripke_Semantics_1_lhs}\\[1ex]
    \Snippet{eq}\\[1ex]
    \Snippet{Intuitionistic_Kripke_Semantics_1_rhs}\\~\\
  \end{center}

  \begin{center}
    \Snippet{Intuitionistic_Kripke_Semantics_2_lhs}\\[1ex]
    \Snippet{eq}\\[1ex]
    \Snippet{Intuitionistic_Kripke_Semantics_2_rhs}\\~\\
  \end{center}

  Here \Snippet{Reflexive_Transitive_Closure} is the \emph{reflexive, transitive closure} of the relation \Snippet{Relation} for \Snippet{Model}
\end{frame}

\subsection{Properties}
\begin{frame}[allowframebreaks]{\insertsectionhead\ \textemdash\
    \insertsubsectionhead}
\emph{Monotony:} \\~\\
\begin{center}
\Snippet{Kripke_model_monotone}\\~\\
\end{center}

\emph{Proof Sketch}: Use induction on \Snippet{phi} while allowing \Snippet{y} to be free.

\framebreak

Monotony, as well as reflexivity and transitivity of \Snippet{Reflexive_Transitive_Closure} give us three other derived rules:

  \begin{center}
    \Snippet{Kripke_models_K}\\[2ex]

    \Snippet{Kripke_models_S}\\[4ex]

    \Snippet{Kripke_models_Modus_Ponens}\\~\\
  \end{center}

These reflect the Combinatory logic typing rules {\sc K\_type}, {\sc S\_type} and {\sc Application\_type} respectively

\end{frame}

\subsection{Soundness}
\begin{frame}[plain]{\insertsectionhead\ \textemdash\
    \insertsubsectionhead}
  \begin{center}
  \Snippet{Combinator_Typing_Kripke_Soundness_alt}\\~\\
  \end{center}

The other direction (called \emph{completeness}) also holds\\~\\

\ldots But the proof is complicated\\~\\

\end{frame}

\subsection{Comonad Refresher}
\begin{frame}[allowframebreaks,fragile]{\insertsectionhead\ \textemdash\
    \insertsubsectionhead}
\begin{lstlisting}
class Comonad w where
  extract :: w a -> a
  duplicate :: w a -> w (w a)
\end{lstlisting}

\framebreak

Comonads obey the laws:\\~\\

\begin{lstlisting}
extract      . duplicate = id
fmap extract . duplicate = id

extract . fmap f
  = f . extract

duplicate . duplicate
  = fmap duplicate . duplicate
\end{lstlisting}

\framebreak

|Cont r| cannot be a monad, because we will show it is impossible to write

\begin{center}
|extract :: ((a -> r) -> r) -> a|
\end{center}

\end{frame}

\subsection{\texttt{extract} Counter Example}
\begin{frame}[allowframebreaks]{\insertsectionhead\ \textemdash\
    \insertsubsectionhead}
  \begin{lemma}
  Let \Snippet{Model} be

  \begin{center}
  \Snippet{Kripke_Cont_Monad_prem_3}
  \end{center}

  where \Snippet{Kripke_Cont_Monad_prem_1} and \Snippet{Kripke_Cont_Monad_prem_2}\\[1ex]

  then
  \begin{center}
    \Snippet{Kripke_Cont_Monad_concl}
  \end{center}
  \end{lemma}

\framebreak

Here's a diagram of what's going on in this model:\\~\\

\begin{center}
\includegraphics[width=0.3\textwidth, angle=0]{kripke_model_counter_example.pdf}
\end{center}

\framebreak

\emph{Proof.}\\~\\

  First observe that \Snippet{Kripke_Cont_Monad_1_a} and
  \Snippet{Kripke_Cont_Monad_1_b}. \\~\\

  Since \Snippet{Kripke_Cont_Monad_1_c}, then
  \begin{center}
    \Snippet{Kripke_Cont_Monad_1} \\~\\
  \end{center}

  \framebreak

  In order to show \Snippet{Kripke_Cont_Monad_2}, we must find a \Snippet{x} such that:\\~\\
  \begin{enumerate}
    \item \Snippet{Kripke_Cont_Monad_2_a}
    \item \Snippet{Kripke_Cont_Monad_2_b}
    \item \Snippet{Kripke_Cont_Monad_2_c}
  \end{enumerate}

   We can see that \Snippet{Kripke_Cont_Monad_2_d} works.\\~\\

   Since all we have is \Snippet{a} and \Snippet{b} to worry about, we have:

\begin{center}
  \Snippet{Kripke_Cont_Monad_2_e}
\end{center}

Hence \Snippet{Kripke_Cont_Monad_2_f} vacuously.

\framebreak

But since \Snippet{Kripke_Cont_Monad_3}, then by modus ponens we have our result!

\hfill\qed

\end{frame}

\subsection{No Combinator For \texttt{extract}}
\begin{frame}[plain]{\insertsectionhead\ \textemdash\
    \insertsubsectionhead}
By the soundness result previously established\\~\\
  \begin{center}
  \Snippet{Combinator_Typing_Kripke_Soundness_alt}\\~\\
  \end{center}

And from the lemma we just proved, if \Snippet{no_extract_prem} then \\~\\

\begin{center}
  \Snippet{no_extract_concl}
\end{center}

\end{frame}


\section{Follow Up}
\begin{frame}[plain]
  \vfill \centering
  \begin{beamercolorbox}[sep=8pt,center,shadow=true,rounded=true]{title}
    \usebeamerfont{title}\insertsectionhead\par%
  \end{beamercolorbox}
  \vfill
\end{frame}

\subsection{\texttt{ContT} Monad Transformer}
\begin{frame}[allowframebreaks,fragile]{\insertsectionhead\ \textemdash\
    \insertsubsectionhead}

\begin{lstlisting}
newtype ContT r m a
  = ContT {runContT :: (a -> m r) -> m r}

type Cont r = ContT r Identity
\end{lstlisting}

\framebreak

|ContT| is thought to \emph{not be} a functor in the category of monads\\~\\

\begin{center}
\emph{Can we prove this?}\\~\\
\end{center}

\framebreak

That is, is there a monad |m| and a |c| such that there is no function:

\begin{lstlisting}
hoist ::
  forall a b.
  (m a -> m b) ->
  (ContT c m a -> ContT c m b)
\end{lstlisting}

which obeys these laws:

\begin{lstlisting}
hoist (f . g) = hoist f . hoist g
hoist id = id
\end{lstlisting}

\begin{center}?\end{center}

\end{frame}


\section{Bibliography}
\begin{frame}{\insertsectionhead}
  \setbeamerfont{bibliography item}{size=\tiny}
  \setbeamerfont{bibliography entry author}{size=\tiny}
  \setbeamerfont{bibliography entry title}{size=\tiny}
  \setbeamerfont{bibliography entry location}{size=\tiny}
  \setbeamerfont{bibliography entry note}{size=\tiny}
  \bibliographystyle{siam}%
  \bibliography{bibliography}
\end{frame}

\end{document}
